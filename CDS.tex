% Tout ce qui est mis derrière un « % » n'est pas vu par LaTeX
% On appelle cela des « commentaires ».  Les commentaires permettent de
% commenter son document - comme ce que je suis en train de faire
% actuellement - et de cacher du code - cf. la ligne \pagestyle.

\documentclass[a4paper]{article}
\usepackage{ dsfont }
% Options possibles : 10pt, 11pt, 12pt (taille de la fonte)
%                     oneside, twoside (recto simple, recto-verso)
%                     draft, final (stade de développement)

\usepackage[utf8]{inputenc}   % LaTeX, comprends les accents !
\usepackage[T1]{fontenc}      % Police contenant les caractères français
\usepackage[francais]{babel}  % Placez ici une liste de langues, la
                              % dernière étant la langue principale

\usepackage[a4paper]{geometry}% Réduire les marges
% \pagestyle{headings}        % Pour mettre des entêtes avec les titres
                              % des sections en haut de page

\title{Valuation CVA sur CDS}           % Les paramètres du titre : titre, auteur, date
\author{Simon Matet \and Antoine Sauvage}
\date{}                       % La date n'est pas requise (la date du
                              % jour de compilation est utilisée en son
			      % absence)

\sloppy                       % Ne pas faire déborder les lignes dans la marge

\begin{document}

\maketitle                    % Faire un titre utilisant les données
                              % passées à \title, \author et \date

\begin{abstract}
Le but de ce document est de modéliser le coût induit sur un contrat de type CDS par le risque de la contrepartie.
\end{abstract}

\tableofcontents              % Table des matières

% \part{Titre}                % Commencer une partie...

\section{Présentation générale}               % Commencer une section, etc.

\subsection{Contexte}         % Section plus petite
Nous appelerons un CDS sans risque un CDS telle que le risque de défaut de la contrepartie est nul, et un CDS avec risque un CDS ou ce risque n'est pas nul.
\subsection{Pricing général}
Le prix d'un CDS est défini par la somme (ici continue) des cash-flows futurs, à leur valeur actualisée.
Dans le cas de cash-flows incertains, c'est leur espérence actualisée qui est prise en compte.
\subsubsection{CDS sans risque}
En cas de faillite de la firme 1 (le sous-jacent du CDS), le taux de recouvrement est de $R_{1}$, et donc la quantité que doit rembourser la firme 2 est de $ (1-R_{1})$.
Le coût est alors de : 
\begin{equation}
(1 - R_{1})\beta (\tau_{1})
\end{equation}
où $\beta(t)$ représente le facteur d'actualisation.
Cet événement n'a lieu que si le temps de défaut $\tau_{1}$ est antérieur à la maturité $T$.
A tout instant $t$, la valeur d'un CDS sans risque est donc de :
\begin{equation}
-\kappa \int_{t}^{\tau_{1} T}\beta(s-t)ds + (1 - R_{1})\beta(\tau_{1})\mathds{1}_{t<\tau_{1}<T}
\end{equation}
ce qui mène, lorsque l'on observe depuis le temps à :





% \subsubsection{Titre}       % Encore plus petite

% \paragraph{Titre}           % Toutes petites sections (le nom \paragraph
                              % n'est pas très bien choisi)

% \subparagraph{Titre}        % La dernière

% \appendix                   % Commençons les annexes

% \section{Titre}             % Annexe A

% \section{Titre}             % Annexe B

% \listoffigures              % Table des figures

% \listoftables               % Liste des tableaux

\end{document}

