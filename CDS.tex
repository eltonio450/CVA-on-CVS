% Tout ce qui est mis derrière un « % » n'est pas vu par LaTeX
% On appelle cela des « commentaires ».Les commentaires permettent de
% commenter son document - comme ce que je suis en train de faire
% actuellement - et de cacher du code - cf. la ligne \pagestyle.


\documentclass[a4paper]{article}
\usepackage[utf8]{inputenc}
\usepackage{amsmath}
\usepackage{amsfonts}
\usepackage{amssymb}
\usepackage{amsthm}
\usepackage{graphicx}

\usepackage[a4paper]{geometry}% Réduire les marges
% \pagestyle{headings}        % Pour mettre des entêtes avec les titres
                              % des sections en haut de page

\title{Credit Valuation Adjustment sur Credit Default Swap}           % Les paramètres du titre : titre, auteur, date
\author{Simon Matet \and Antoine Sauvage}
\date{}                       % La date n'est pas requise (la date du
                              % jour de compilation est utilisée en son
			      % absence)

\sloppy                       % Ne pas faire déborder les lignes dans la marge

\begin{document}

\maketitle                    % Faire un titre utilisant les données
                              % passées à \title, \author et \date

\begin{abstract}
Le Credit Default Swap (CDS) est un type de contrat permettant à un créancier de s'assurer contre le risque de défaut d'un débiteur.
 
Cependant, l'institution émettrice d'un CDS peut elle même faire défaut avant ou en même temps que le débiteur.
 La prise en compte de cette possibilité dans le \textit{pricing} d'un CDS et les stratégies pour se prémunir de ce risque de contrepartie font l'objet de ce rapport.

\end{abstract}

\tableofcontents              % Table des matières

% \part{Titre}                % Commencer une partie.
.
.


\section{Présentation générale}               % Commencer une section, etc.


\subsection{Contexte et notations}         % Section plus petite
Nous appellerons un CDS sans risque un CDS tel que la probabilité de défaut de la contrepartie est nulle et un CDS avec risque un CDS où cette probabilité est strictement positive.
 Dans le cadre de ce rapport, une firme 0 détient une unité de dette à risque de la firme 1, qui peut faire défaut.
 Dans ce cas la firme 1 rembourse ses dettes à hauteur de $R_{1} \in \left[ 0, 1 \right[$.
 Pour se couvrir contre ce risque, la firme 0 achète un CDS à la firme 2 et s'engage à verser une prime $\kappa$ en échange d'une assurance contre le défaut de 1 jusqu'à la date de maturité $T$.
 La valeur du contrat dépend cependant de la probabilité que l'assureur 2 fasse défaut avant son terme.
\\ \\
Nous noterons donc :\\
$\beta (t)$ Le taux d'actualisation\\
$T$ La date de maturité du contrat\\
$R_{1}$ Le taux de recouvrement de la firme 1\\
$R_{2}$ Le taux de recouvrement de la firme 2\\
$\tau_{1} (\omega)$ Temps aléatoire du défaut de la firme 1\\
$\tau_{2} (\omega)$ Temps aléatoire du défaut de la firme 2\\
$\kappa$ La prime versée par la firme 0 à la firme 2 à chaque unité de temps\\
$p (t, \omega)$ Le cash flow à venir après la date $t$ pour une réalisation $\omega$ de la firme 2 vers la firme 0 en supposant que 2 ne fasse jamais défaut\\
$\pi (t, \omega)$ Le cash flow à venir après la date $t$ pour une réalisation $\omega$ de la firme 2 vers la firme détentrice du CDS dans le cas où 2 peut faire défaut\\
$P (t, \omega) = \mathbb{E}_{t} [p]$ le cash-flow espéré conditionnellement à l'information connue à la date t, sans risque de contrepartie\\
$\Pi (t, \omega) = \mathbb{E}_{t} [\pi]$ le cash-flow espéré conditionnellement à l'information connue à la date t, avec risque de contrepartie


\subsection{Analyse des cash-flows}
Dans le cadre d'un marché sans possibilités d'arbitrage, le prix d'un CDS à la date t est défini par l'espérance des cash-flows futurs actualisés.

\subsubsection{Cash-flow pour un CDS sans risque}
À la date $s$, la firme 0 doit verser à 2 la prime $\kappa$.
 Si de plus $\tau_1 = s$, alors 1 rembourse uniquement $R_1$ de sa dette et 2 verse donc les $1-R_1$ restant à 0.
 En remarquant que le coefficient d'actualisation entre $t\leq s$ est $\frac{\beta_{s}}{\beta_{t}}$, le cash-flow lié au CDS à la date s actualisé pour la date t est donc : 
\begin{equation*}
\frac{- \beta(s) \kappa+ \beta(\tau_{1}) \left( 1 - R_{1} \right)  \mathbf{ 1 }_{\tau_{1} } \left( s \right) }{\beta (t)} 
\end{equation*}
En intégrant entre t et $T \wedge \tau_{1}$ on trouve la formule du cash-flow futur à partir de la date t (il s'agit par exemple du prix maximal qu'un acheteur omniscient serait prêt à payer pour le contrat à la date t).

\begin{equation}
\beta (t) p(t) = -\kappa \int\limits_{t}^{T \wedge \tau_{1}} \beta(s)ds + \beta(\tau_{1})\left( 1 - R_{1} \right)\mathbf{ 1 }_{\left] t, T \right[}(\tau_{1})
\end{equation}
\subsubsection{Cas d'un CDS risqué}
On reprend la précédente analyse avec cette fois deux nouveaux termes.
 Le premier correspond au défaut de 1 et 2 en même temps.
 Dans ce cas, 2 devrait rembourser $1-R_{1}$.
 Suite à son défaut, 2 ne rembourse que $R_{2}(1-R_{1})$, d'où un terme en $\mathbf{1}_{\tau_{1} = \tau_{2} = s} (1-R_{2})(1-R_{1})$.
 Par ailleurs, si seul 2 fait défaut avant 1, soit le contrat a un prix positif à cet instant et 2 doit donc a 0 $P_{s}^{+}$ mais ne lui rembourse que $R_{2}P_{s}^{+}$ car il est en défaut.
 Sinon, 0 rembourse la valeur du contrat $-P_{s}^{-}$.
 En sommant jusqu'à la fin du contrat, ie.
 $T \wedge \tau_{1} \wedge \tau_{2}$, on trouve :
\begin{multline}
\beta(t)\pi(t) = -\kappa \int\limits_{t}^{T \wedge \tau_{1} \wedge \tau_{2}} \beta(s)ds + \beta\left(\tau_{1}\right)\left( 1 - R_{1} \right)\mathbf{ 1 }_{\left] t, T \wedge \tau_{2} \right[}(\tau_{1}) \\
+ \beta(\tau_{1}) R_{2} \left( 1 - R_{1} \right) \mathbf{1}_{t < \tau_{1} = \tau_{2} < T} 
+ \beta(\tau_{2}) \left( R_{2}P_{\tau_{2}}^{+} - P_{\tau_{2}}^{-} \right) \mathbf{1}_{t < \tau_{2} < T \wedge \tau_{1}}
\end{multline}
\pagebreak

\section{Modèle}
\subsection{Hypothèses}
On modélise la santé financière des entreprises $i, i \in \lbrace 1, 2 \rbrace$ par deux processus d'Itô corrélés, 
\begin{eqnarray}
dX^{1}_{t} &=& b_{1}(t)dt + \sigma_{1}(t)dW_{t}^{1} \\
dX^{2}_{t} &=& b_{2}(t)dt + \sigma_{2}(t)\left(\rho dW_{t}^{1} + \sqrt{1 - \rho^{2}} dW_{t}^{2}\right)
\end{eqnarray}
Où $W_{1}$ et $W_{2}$ sont deux mouvements browniens indépendants et $\rho$ la corrélation entre les deux processus d'Itô (on vérifie sans peine que $\rho dW^{1} + \sqrt{1 - \rho^{2}} dW^{2}$ donne un mouvement brownien de covariance $\rho$ avec $W_{1}$, en calculant les fonctions caractéristiques des accroissements par exemple). On notera $\mathbb{X}$ la filtration canoniquement associée à $(X^{1}, X^{2})$. On suppose de plus que les processus $b_{i}, \sigma_{i}$ sont $\mathbb{X}$ adaptés et vérifient les conditions d'intégrabilité usuelles ($b_{i}$ intégrables, $\sigma_{i}$ de carré intégrable presque sûrement). \\ \\
On introduit ensuite trois variables de défaut qui sont les temps d'arrêt $d^{1}$, temps de défaut endogène de la firme 1, $d^{2}$, temps de défaut endogène de la firme 2 et $d^{3}$, temps de défaut commun exogène. On suppose que ce sont des variables exponentielles telles que :
\begin{eqnarray}
\mathbb{P}\left( d^{i} > t \mid \mathbb{X}_{t} \right) &=& e ^{ -\int \limits_{0}^{t} l_{i} (t, X^{i}_{s})ds}\,,\qquad  i = 1, 2 \\
\mathbb{P}\left( d^{3} > t \right) &=&  e ^{ -\int \limits_{0}^{t} l_{3} (t)ds}
\end{eqnarray}
Avec la particularité que $l_{3}$ est une fonction déterministe qui ne dépend pas des $X^{(1,2)}$ et que $\lbrace d^{3} < t \rbrace$ est indépendant de $\mathbb{X}_{t} \vee \sigma(d^{1}) \vee \sigma(d^{2})$.
On suppose en outre que les temps de défaut endogènes sachant $\mathbb{X}$ sont indépendants, ie. 
\begin{equation*}
\mathbb{E} \left(d^1 \mid \mathbb{X}_t \right) \perp \mathbb{E} \left(d^2 \mid \mathbb{X}_t \right) \,,\qquad  \forall t > 0
\end{equation*}
Les temps de défaut réellement observés pour chaque firme seront donc $\tau^{1} = d^{1} \wedge d^{3}$ et  $\tau^{2} = d^{2} \wedge d^{3}$. On utilisera en outre les fonctions indicatrices de défaut :
\begin{equation*}
H^{i}_{t} = \mathbf{1}_{\left[0, t \right]} (\tau^{i})
\end{equation*}
Et $\mathbb{H}$ la filtration associée. Enfin, on posera $\mathbb{F} = \mathbb{X} \vee \mathbb{H}$. \\ \\

\subsection{Générateurs infinitésimaux et premières propriétés}
On peut maintenant calculer le générateur du processus aléatoire $(X, H) = (X_{1}, X_{2}, H_{1}, H_{2})$. Pour $u \in \mathcal{C}^{1, 2}$ ($\mathcal{C}^{1}$ pour la première variable, $\mathcal{C}^{2}$ pour la seconde variable et sans régularité pour la troisième qui est dans un espace discret de toute façon), on a
\begin{equation*}
\mathcal{A}u : (x,t)
\rightarrow \space \lim_{h\downarrow 0} \frac {\mathbb{E}^{x}(u(t+h, X_{t+h}, e_{t+h}) - u(t,x,e)}{h}
\end{equation*}
\begin{multline*}
\mathcal{A}u(t,x,e) = \lim_{h\downarrow 0} \frac {\mathbb{E}^{x}(u(t+h, X_{t+h}, e_{t+h}) - u(t,x,e)}{h} \\
= \lim_{h\downarrow 0} \mathbb{E}\left[
\mathbb{E} \left[ \frac{u (t+h, X_{t+h}, e_{t+h}) - u(t+h, X_{t+h}, e_{t})}{h} \mid \mathbb{X}_{t+h} \vee \mathbb{F}_{t}  \right] \mid X_t = x, \, H_t = e \right] \\
 + \mathbb{E} \left[ \frac {u(t+h, X_{t+h}, e_{t}) -  u(t, X_{t}, e_{t})}{h} \mid X_t = x, \, H_t = e \right]
\end{multline*}
Par conditionnement itéré. On décompose le premier terme en :
\begin{multline*}
\lim_{h\downarrow 0} \mathbb{E} \left[
\mathbb{E} \left[ \frac{u (t+h, X_{t+h}, e_{t+h}) - u(t+h, X_{t+h}, e_{t})}{h} \mid \mathbb{X}_{t+h} \vee \mathbb{F}_{t}  \right] \mid X_t = x, \, H_t = e \right] \\
=  \lim_{h\downarrow 0} \mathbb{E} \left[ \sum \limits_{\epsilon \in \lbrace 0, 1 \rbrace^{2}} \mathbb{E} \left[ \frac{u (t+h, X_{t+h}, \epsilon) - u(t+h, X_{t+h}, e)}{h} \mathbf{1}_{e_{t+h} = \epsilon} \mid \mathbb{X}_{t+h} \vee \mathbb{F}_{t}  \right] \vert X_t = x, \, H_t = e \right]
\end{multline*}
Au plus trois transitions entre $\epsilon$ et e son possibles : défaut de 1, défaut de 2 et défaut simultanée. On posera, pour $e  \in \lbrace 0, 1 \rbrace^{2}$, $e^{i}$ le vecteur dont on a remplacé la i-ième coordonnée par 1. Alors, si $e \neq e^{i}$
\begin{eqnarray*}
\mathbb{P} \left(e_{t+h} = e^{i} \mid \mathbb{X}_{t+h} \vee \mathbb{F}_{t}  \right) = 1 - e^{ -\int \limits_{t}^{t+h} l_{i} (X^{i}_{s}, s)ds} + o(h)
\end{eqnarray*}
Le o(h) rendant compte de deux transitions successives (la probabilité d'un tel événement est en $\mathcal{O} (h^2)$). On trouve le même type d'égalité pour le défaut commun et, en remarquant que pour $e = e^{i}$ la différence des fonctions u est de toute façon nulle, en faisant tendre h vers 0 et par continuité du processus d'Itô qui vaut x en t, on trouve que le premier terme vaut :
\begin{equation}
T_{1} = \sum \limits_{1 \leqslant i \leqslant 2} l_{i} (t, x_{i}) (u(t, x, e^{i}) - u(t, x, e)) + l_{3}(t)(u(t,x,(1,1)) - u(t,x,e))
\end{equation}
Le second terme se calcule grâce à la formule d'Itô :
\begin{multline*}
u(t+h, X_{t+h}, e) - u(t, X_{t}, e) = \int \limits_{t}^{t+h} Du(s,X_{s},e)\cdot dX_{s} +\\ \int \limits_{t}^{t+h} \left( u_{t}(s,X_{s}, e) + \frac{1}{2} (\sigma_{1}^{2} \partial_{x_1^2}^2u(s, X_{s}, e) + \sigma_{2}^{2} \partial_{x_2^2}^2u(s, X_{s}, e) + 2\rho\sigma_{1}\sigma_{2}\partial_{x_1 x_2}^2 u(s, X_{s}, e)) \right) ds
\end{multline*}
Encore une fois, par continuité des différentes fonctions (hypothèse sur u à expliciter, il faut que $Du(t, X_t, e) \in \mathbb{H}^2$), on trouve finalement
\begin{multline}
\mathcal{A}u(t,x,e) = \partial_{t}  u(t,x,e) + \sum \limits_{1\leq i\leq2}l_i(t,x_i)(u(t,x,e^i) - u(t,x,e)) + l_3(t)(u(t,x,(1,1)) - u(t,x,e))\\
 + \sum \limits_{1\leq i\leq2} \left( b_i(t,x_i)\partial_{x_i} u(t,x,e) + \frac{1}{2}\sigma_i^2(t,x_i)\partial^2_{x_i^2}u(t,x,e) \right) + \rho \sigma_1(t,x_1)\sigma_2(t,x_2)\partial^2_{x_1,x_2}u(t,x,e) 
\end{multline}

\section{Pricing}
\subsection{CDS non risqué}
Le prix du CDS étant donné par l'espérance du cash-flow actualisé, on cherche à calculer $P_{t} = \mathbb{E}_{t} \left[ p^{t} \right]$ où $p^{t}$ est donné par l'équation (1). Pour cela, on sépare le processus $P_{t}$ en une intégrale par rapport au temps et un terme d'espérance nulle.
Ainsi, \\
\begin{eqnarray*}
\beta(t) p^t =& -\kappa \int \limits_{t}^{T}\beta(s)(1-H^{1}_{s})ds + (1-R_{1})\int \limits_{t}^{T} \beta(s) dH_{s}^{1} \\
%\text{Où le dernier terme est, pour} \omega \text{fixé, l'intégrale de Stieltjes par rapport à la fonction croissante} H^{1}_{s}. \\
=&\int \limits_{t}^{T}\beta(s)(1-H^{1}_{s})p(s, X^{1}_s)ds + (1-R_{1})\int \limits_{t}^{T} \beta(s) dM_{s}^{1} \\
\end{eqnarray*}
En introduisant
\begin{eqnarray*}
p(s, x) :&=& (1-R_1)q_1(t,x) - \kappa \\
M_t^i :&=& H_t^i - \int \limits_0^t q_i (s, X_s, H_s) ds
\end{eqnarray*}
Or, le processus $M_t^i$ ainsi défini est une martingale puisque, dans le cas général, $u(t, X^i_t, H^i_t) - \int \limits_0^t \mathcal{A}u(s, X^i_s, H^i_s)ds$ est une martingale pour $u \in \mathcal{C}^{1, 2}$ à dérivées et sauts bornés. \\
En effet,
\begin{multline*}
d_t\mathbb{E}_s \left[ u(t, X^i_t, H^i_t) - \int \limits_0^t \mathcal{A}u(r, X^i_r, H^i_r)dr \right] \\
= \lim \limits_{h \rightarrow 0} \mathbb{E}_s \left[ \frac{\mathbb{E}_t (u(t+h, X^i_{t+h}, H^i_{t+h})) - u(t, X^i_t, H^i_t)}{h} - \mathcal{A}u(t,X^i_t,H^i_t)\right]
= 0
\end{multline*}
Par conditionnements itérés et en intervertissant limite et intégrale sous les hypothèses appropriées La formule pour le générateur aléatoire montre alors qu'il est borné et qu'on peut intervertir pour le terme avec l'intégrale. De même, les formules de 2.2 permettent de montrer que le premier terme est borné. En appliquant le résultat à la projection sur la dernière variable, on obtient bien que $M^i_t$ est une martingale. \\ \\
Il suit par un théorème d'intégration par rapport à une martingale bornée que l'espérance en t du second terme de (8) est nulle et donc :
\begin{equation*}
\mathbb{E}_{t} \left[ \beta(t)p_{t} \right] =  \mathbb{E}_{t} \left[ \int \limits_{t}^{T}\beta(s)(1-H^{1}_{s})p(s, X^{1}_s)ds \right]
\end{equation*}
Or, le second terme de cette équation s'annule quand $H^i_t = 1$. On peut donc définir la variable aléatoire
\begin{equation*}
V = \frac{\mathbb{E}_{t} \left[ \int \limits_{t}^{T}\beta(s)(1-H^{1}_{s})p(s, X^{1}_s)ds \right]}{\beta(t)(1- H^i_t)}
\end{equation*}
Avec $V=0$ si $H^i_t = 1$. Par la propriété de Markov de X et H, V est $\sigma(X_t^i,\, H^i_t)$ mesurable, ie. $V = f(t,X^i_t, H^i_t)$ En posant $v(t, X^i_t) = f(t,X^i_t, 0)$ on obtient 
\begin{equation*}
\beta(t)P_{t} =  \mathbb{E}_{t} \left[ \int \limits_{t}^{T}\beta(s)(1-H^{1}_{s})p(s, X^{1}_s)ds \right] = (1 - H^i_t)\beta(t)v(t,X_t^i)
\end{equation*}
Par ailleurs, si v est suffisamment différentiable, on trouve en calculant le générateur infinitésimal des deux membres et grâce au résultat de 2.2 :
\begin{equation}
\left\{
\begin{split}
v(T, x) = 0,\ x \in \mathbb{R} \\
\left(\partial_t + b_i(t,x_i)\partial_{x_i} + \frac{1}{2}\sigma_i^2(t,x_i)\partial^2_{x^2_i}\right)v(t,x_i) - (r+q_i(t,x_i))v(t,x_i) + p(t,x_i) =0, \ t\in\left[0,T\right[,\ x_i \in \mathbb{R}
\end{split}
\right.
\end{equation}
Sous certaines hypothèses de régularité et de croissance (à expliciter), on en déduit par le théorème de Feynman-Kac :
\begin{equation}
v(t,x) = \mathbb{E} \left[ \int \limits_t^T e^{-\int \limits_{t}^{s} (r + q_i(r, X^i_r))dr} p(s, X^i_s)ds \mid X^i_t = x\right]
\end{equation}

\subsection{CDS risqué}
(parallèle rapide)
%Le même cheminement donne, en posant
%\begin{equation}
%\pi (t,x) = (1-R_1) \left[ l_1 (t,x_1) + R_2 l_3(t) \right] + l_2 %\left[ R_2 v^{+}(t,x_1) - v^{-}(t,x_1) \right] - \kappa
%\end{equation}
% \subsubsection{Titre}      % Encore plus petite

% \paragraph{Titre}           % Toutes petites sections (le nom \paragraph
                              % n'est pas très bien choisi)

% \subparagraph{Titre}        % La dernière

% \appendix                   % Commençons les annexes

% \section{Titre}             % Annexe A

% \section{Titre}             % Annexe B

% \listoftables               % Liste des tableaux

\end{document}


